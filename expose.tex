\documentclass[a4paper,12pt,oneside, bibliography=totoc]{scrartcl}

% ------ Schrift und -kodierung & Sprache ------
\usepackage[TS1,T1]{fontenc}
\usepackage{lmodern}
\usepackage{fourier,erewhon}
\usepackage[german]{babel}

% ------ Mathe & Tabellen & Grafiken ------
\usepackage{amsmath, amssymb}

\usepackage{array, booktabs, longtable}
\usepackage{tabularx}

\usepackage{graphicx}
\usepackage[x11names, table]{xcolor}

\usepackage{enumitem}
\usepackage{listings}
\usepackage[htt]{hyphenat} %damit texttt noch Linebreaks mit Silbentrennung erzeugt
\newcommand{\code}[1]{\texttt{#1}} %Programmcode im Textfluss in passendem Font ausgeben

\usepackage{float} 
\usepackage{pbox}
\usepackage{algorithm}
\usepackage{siunitx}

 
% ------ Abbildungen & Ablürzungen ------
\usepackage[printonlyused]{acronym}
\usepackage{caption}
\usepackage{subcaption}
%\usepackage[page, title, titletoc, header]{appendix}

   
% ----- Literatur ------
\usepackage{csquotes}
\usepackage[
   backend=biber,
   sorting= none,
   firstinits=true,
   date=long,
   urldate=long
]{biblatex}
\addbibresource{literatur1.bib}

% ------ Hyperlinks ------
\usepackage[]{hyperref}
\usepackage{cleveref}


% ------ Zeitplan ------
\newcommand{\foo}{\color{Firebrick3}\raisebox{-0.2ex}{\scalebox{1.5}{\textbullet}}\hskip -4pt\vrule width 1pt\hspace{\labelsep}}

\newcommand{\bfoo}{\raisebox{2.1ex}[0pt]{\makebox[\dimexpr2\tabcolsep]%
	{\color{Firebrick3}\tiny\textbullet}}}%
\newcommand{\tfoo}{\makebox[\dimexpr2\tabcolsep]%
	{\color{Firebrick3}\scalebox{1.8}{$\downarrow$}\hskip 0.9pt }}

% https://markov.htwsaar.de/tex-archive/macros/latex/contrib/xcolor/xcolor.pdf	
% https://tex.stackexchange.com/questions/196794/how-can-you-create-a-vertical-timeline
% https://tex.stackexchange.com/questions/466452/margin-in-title-page



\begin{document}

	\titlehead{
	\begin{flushleft}
	   \vspace{-2.0cm}
	   \includegraphics[width=5cm]{figures/logo.png}\\
	   \smallskip
	   \hrule
	   \vspace{0.04cm}\hspace*{2cm} Institute of Computer Science,    Computer Engineering Group \par
	   \vspace{0.06cm}\hrule
	\end{flushleft}
	}

	\subject{Expos\'e einer Bachelorthesis}
	\title{\vspace{-0.8cm} Titel}
	\author{Name Vorname,  Matrikelnr: 1000000}
	\date{\today}
	%Advisor:  \\ %Deutsch: Erstebetreuer
	%Co-Advisor:  %Deutsch: Zweitbetreuer

	\maketitle
	
	\section{Einleitung}\label{sec:problemstellung}

Problemstellung: Welches konkrete Problem wird adressiert?\\
Motivation: Warum ist dieses Problem relevant?\\
Kontext: Wissenschaftlich, gesellschaftlich oder industriell? \\

\section{State of the Art}

\begin{itemize}
    \item Wissenschaftlicher / Technologischer Hintergrund
    \item Überblick über den Stand der Technik, den die Arbeit ergänzt, verbessert \\ (Relevante Theorien / Methoden)
    \item Wichtigste Arbeiten erwähnen
    \item Genutzte Daten, Tools, Systeme
\end{itemize}


\section{Ziel}

\begin{itemize}
    \item Abgrenzung der eigenen Arbeit
    \item Erwarteter wissenschaftlicher / praktischer Beitrag 
    \item Es sollte den Stand der Technik verbessern, neue empirische Daten liefern oder zu neuem Ergebnis führen
    \item Art der Arbeit:\\
    z.B.: Analytisch, Empirisch, Konstruktiv, Literaturarbeit, Kombination
\end{itemize}

 

\section{Zeitplan}

In Wochen, Monate planen \\
Unmittelbare Zukunft detaillierter planen \\

\renewcommand\arraystretch{1.4}\arrayrulecolor{Firebrick3}
\begin{longtable}{@{\,}r <{\hskip 2pt} !{\foo} >{\raggedright\arraybackslash}p{13cm}}
    \caption{Timeline} \\[-0.5ex]
    \toprule
    \addlinespace[1.5ex] 
    \multicolumn{1}{c!{\bfoo}}{}& \\[-2.3ex]

           - 21.04 & \textbf{Vorbereitung und Literaturrecherche}
                     \begin{enumerate}[noitemsep]%[itemsep=1cm,parsep=0.5cm]
                            \item[\textbullet] Relevante wissenschaftliche Quellen und Literatur sammeln
                            \item[\textbullet] Theoretischer Hintergrund der Arbeit entwickeln
                            \item[\textbullet] Analyse des State of the Art
                            \item[\textbullet] Präzisierung der Problemstellung und Zieldefinition    
                        \end{enumerate} \\

        22.04 - 05.05 & \textbf{Methodik}
                        \begin{enumerate}[noitemsep]
                            \item[\textbullet] Geeignete Methoden und Datenstrukturen auswählen
                            \item[\textbullet] Detaillierten Implementierungsplan erstellen
                        \end{enumerate} \\

        Woche 6  & \textbf{Implementierung \& erste Tests}
                        \begin{enumerate}[noitemsep]
                                \item[\textbullet] Implementierung der Kernkomponenten
                                \item[\textbullet] Erste Tests durchführen
                                \item[\textbullet] 
                        \end{enumerate} \\

        Woche 7 -8 & \textbf{Erweiterung \& Evaluation} 
                        \begin{enumerate}[noitemsep]
                            \item[\textbullet] Integration weiterer Komponenten
                            \item[\textbullet]
                        \end{enumerate}\\

        Woche  & \textbf{Analyse}
                        \begin{enumerate}[noitemsep]
                            \item[\textbullet] Interpretation 
                            \item[\textbullet] 
                            \item[\textbullet] 
                        \end{enumerate}\\

        Monat x & \textbf{Abschluss}
                        \begin{enumerate}[noitemsep]
                            \item[\textbullet] 
                            \item[\textbullet] 
                            \item[\textbullet] 
                        \end{enumerate}\\

    \multicolumn{1}{c!{\tfoo}}{}& \\[-2.3ex]
\end{longtable}


\textbf{ } \\
\newline Weitere Quellen die ich für meine Arbeit verwenden werde: \cite{model}
	%\ac{Abk.}         % fügt die Abkürzung ein, außer beim ersten Aufruf, hier wird die Erklärung mit angefügt
%\acs{Abk.}        % fügt die Abkürzung ein
%\acf{Abk.}        % fügt die Abkürzung UND die Erklärung ein
%\acl{Abk.}        % fügt nur die Erklärung ein

\section*{Acronyms}

\textit{Nur erforderlich, wenn im Exposé Abkürzungen verwendet werden.}

%%%%%%%%%%%%%%%%%%%%%%%
\begin{acronym}[E/E/PE] %sorgt fur indention
	\acro{IDE}{\emph{Integrated Development Environment}}
	\acro{JVM}{\emph{Java Virtual Machine}}
	\acro{UML}{\emph{Unified Modeling Language}}
\end{acronym}

	%Prints references
	\printbibliography
\end{document}

