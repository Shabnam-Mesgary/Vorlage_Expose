\section{Einleitung}\label{sec:problemstellung}

Problemstellung: Welches konkrete Problem wird adressiert?\\
Motivation: Warum ist dieses Problem relevant?\\
Kontext: Wissenschaftlich, gesellschaftlich oder industriell? \\

\section{State of the Art}

\begin{itemize}
    \item Wissenschaftlicher / Technologischer Hintergrund
    \item Überblick über den Stand der Technik, den die Arbeit ergänzt, verbessert \\ (Relevante Theorien / Methoden)
    \item Wichtigste Arbeiten erwähnen
    \item Genutzte Daten, Tools, Systeme
\end{itemize}


\section{Ziel}

\begin{itemize}
    \item Abgrenzung der eigenen Arbeit
    \item Erwarteter wissenschaftlicher / praktischer Beitrag 
    \item Es sollte den Stand der Technik verbessern, neue empirische Daten liefern oder zu neuem Ergebnis führen
    \item Art der Arbeit:\\
    z.B.: Analytisch, Empirisch, Konstruktiv, Literaturarbeit, Kombination
\end{itemize}

 

\section{Zeitplan}

In Wochen, Monate planen \\
Unmittelbare Zukunft detaillierter planen \\

\renewcommand\arraystretch{1.4}\arrayrulecolor{Firebrick3}
\begin{longtable}{@{\,}r <{\hskip 2pt} !{\foo} >{\raggedright\arraybackslash}p{13cm}}
    \caption{Timeline} \\[-0.5ex]
    \toprule
    \addlinespace[1.5ex] 
    \multicolumn{1}{c!{\bfoo}}{}& \\[-2.3ex]

           - 21.04 & \textbf{Vorbereitung und Literaturrecherche}
                     \begin{enumerate}[noitemsep]%[itemsep=1cm,parsep=0.5cm]
                            \item[\textbullet] Relevante wissenschaftliche Quellen und Literatur sammeln
                            \item[\textbullet] Theoretischer Hintergrund der Arbeit entwickeln
                            \item[\textbullet] Analyse des State of the Art
                            \item[\textbullet] Präzisierung der Problemstellung und Zieldefinition    
                        \end{enumerate} \\

        22.04 - 05.05 & \textbf{Methodik}
                        \begin{enumerate}[noitemsep]
                            \item[\textbullet] Geeignete Methoden und Datenstrukturen auswählen
                            \item[\textbullet] Detaillierten Implementierungsplan erstellen
                        \end{enumerate} \\

        Woche 6  & \textbf{Implementierung \& erste Tests}
                        \begin{enumerate}[noitemsep]
                                \item[\textbullet] Implementierung der Kernkomponenten
                                \item[\textbullet] Erste Tests durchführen
                                \item[\textbullet] 
                        \end{enumerate} \\

        Woche 7 -8 & \textbf{Erweiterung \& Evaluation} 
                        \begin{enumerate}[noitemsep]
                            \item[\textbullet] Integration weiterer Komponenten
                            \item[\textbullet]
                        \end{enumerate}\\

        Woche  & \textbf{Analyse}
                        \begin{enumerate}[noitemsep]
                            \item[\textbullet] Interpretation 
                            \item[\textbullet] 
                            \item[\textbullet] 
                        \end{enumerate}\\

        Monat x & \textbf{Abschluss}
                        \begin{enumerate}[noitemsep]
                            \item[\textbullet] 
                            \item[\textbullet] 
                            \item[\textbullet] 
                        \end{enumerate}\\

    \multicolumn{1}{c!{\tfoo}}{}& \\[-2.3ex]
\end{longtable}


\textbf{ } \\
\newline Weitere Quellen die ich für meine Arbeit verwenden werde: \cite{model}